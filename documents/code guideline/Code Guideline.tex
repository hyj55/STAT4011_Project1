\documentclass[12pt,a4paper]{article}
\usepackage{blindtext}
\usepackage[T1]{fontenc}
\usepackage[utf8]{inputenc}
\usepackage{amsmath}
\usepackage{underscore}
\usepackage{hyperref}
\usepackage{geometry}
\usepackage{bbm}
\hypersetup{hidelinks,
	colorlinks=true,
	allcolors=black,
	pdfstartview=Fit,
	breaklinks=true}

\usepackage{enumerate,authblk,indentfirst,authblk,ragged2e}
\usepackage{graphicx} 
\usepackage{float}
\usepackage{subfigure} 
\setlength{\parskip}{0.5em} 
\setlength{\parindent}{0em}
\geometry{a4paper,left=1.8cm,right=1.8cm,top=2.2cm,bottom=2.2cm}
\title{A Guideline for Playing with Code}

\author{DAI Qiyu 1155141616}
{\tiny {\tiny {\scriptsize {\scriptsize {\tiny }}}}}



\begin{document}
	\maketitle
To make it convenient to replicate our results, three ipynb files and their accompanying dataset are uploaded separately, so that pieces of code can be easily fixed or improved.

\begin{itemize}
	\item \textbf{data$\_$collection.ipynb and Lianjia$\-$raw.csv} \\
	This file tries to  automatically scrap data from the \textit{Lianjia} website. The output is raw data.csv.
	\item \textbf{data$\_$preprocessing.ipynb and Lianjia$\_$transformed.csv} \\
	After importing raw data.csv, the data$\-$preprocessing.ipynb conducts data processing and the output is data$\_$transformed.csv.
	\item \textbf{model.ipynb} \\
	It does regression analysis using the dataset data$\_$transformed.csv.
\end{itemize}


You are kindly reminded that, to run the code on your computer, the working directory should be reset correspondingly.
\end{document}
